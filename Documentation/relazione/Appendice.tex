% -------------------------- APPENDICI ------------------------------------------

\appendix
\chapter{Guida utente}

\begin{figure}[H]
	\centering{}
	\includegraphics[width=1\linewidth]{../screens/bullet_ballet_screen0.png}
	%\caption{Bullet Ballet}
	\label{img:bullet_ballet_screen0}
\end{figure}

\subsection{Modalità d'uso}

\textsf{\small Per avviare l'applicazione occorre eseguire il comando \emph{./gradlew run}.} \\

\subsection{Requisiti}

\begin{itemize}
	\item \textsf{\small JDK 11+}
	\item \textsf{\small Gradle 8+}
\end{itemize}

\subsection{Comandi del gioco}

\begin{itemize}
	\item \textsf{\small freccia su : saltare}
	\item \textsf{\small freccia destra : andare a destra}
	\item \textsf{\small freccia sinistra : andare a sinistra}
	\item \textsf{\small spazio : sparare}
\end{itemize}

\begin{figure}[H]
	\centering{}
	\includegraphics[width=1\linewidth]{../screens/bullet_ballet_screen3.png}
	%\caption{}
	\label{img:bullet_ballet_screen3}
\end{figure}

\begin{comment}
\begin{figure}[H]
	\centering{}
	\includegraphics[width=1\linewidth]{../screens/bullet_ballet_screen5.png}
	%\caption{}
	\label{img:bullet_ballet_screen5}
\end{figure}
\end{comment}

\subsection{Impostazioni di Gioco}

\textsf{\small Attraverso il menù di gioco è possibile modificare le impostazioni: } \\

\begin{itemize}
	\item \textsf{\small Risoluzione}
	\item \textsf{\small Livello di difficoltà}
	\item \textsf{\small Volume}
	\item \textsf{\small Lingua:}
	\begin{itemize}
		\item \textsf{\small Inglese}
		\item \textsf{\small Italiano}
		\item \textsf{\small Spagnolo}
		\item \textsf{\small Tedesco}
		\item \textsf{\small Polacco}
		\item \textsf{\small Russo}
		\item \textsf{\small Arabo Standard Moderno}
		\item \textsf{\small Francese}
		\item \textsf{\small Giapponese}
	\end{itemize}
\end{itemize}

\begin{figure}[H]
	\centering{}
	\includegraphics[width=1\linewidth]{../screens/bullet_ballet_screen2.png}
	%\caption{}
	\label{img:bullet_ballet_screen2}
\end{figure}

\begin{comment}
\begin{figure}[H]
	\centering{}
	\includegraphics[width=1\linewidth]{../screens/bullet_ballet_screen6.png}
	%\caption{}
	\label{img:bullet_ballet_screen6}
\end{figure}
\end{comment}

\chapter{Esercitazioni di laboratorio}

\textsf{\small Qui di seguito, i links di alcuni esercizi che abbiamo svolto:} \\

\section*{Esempio}

\subsection{Alessandro Pioggia}

\begin{itemize}
	\item Laboratorio 04: \url{https://virtuale.unibo.it/mod/forum/discuss.php?d=62685#p101507}
	\item Laboratorio 05: \url{https://virtuale.unibo.it/mod/forum/discuss.php?d=62684#p101217}
	\item Laboratorio 06: \url{https://virtuale.unibo.it/mod/forum/discuss.php?d=62579#p100880}
	\item Laboratorio 07: \url{https://virtuale.unibo.it/mod/forum/discuss.php?d=62582#p100893}
	\item Laboratorio 08: \url{https://virtuale.unibo.it/mod/forum/discuss.php?d=63865#p107314}
	\item Laboratorio 09: \url{https://virtuale.unibo.it/mod/forum/discuss.php?d=64639#p103992}
	\item Laboratorio 10: \url{https://virtuale.unibo.it/mod/forum/discuss.php?d=66753#p106887}
\end{itemize}

\subsection{Leon Baiocchi}

\begin{itemize}
	\item Laboratorio 04: \url{https://virtuale.unibo.it/mod/forum/discuss.php?d=62685#p101274}
	\item Laboratorio 05: \url{https://virtuale.unibo.it/mod/forum/discuss.php?d=62684#p101286}
	\item Laboratorio 06: \url{https://virtuale.unibo.it/mod/forum/discuss.php?d=62579#p100978}
	\item Laboratorio 07: \url{https://virtuale.unibo.it/mod/forum/discuss.php?d=62582#p101496}
	\item Laboratorio 08: \url{https://virtuale.unibo.it/mod/forum/discuss.php?d=63865#p103119}
	\item Laboratorio 09: \url{https://virtuale.unibo.it/mod/forum/discuss.php?d=64639#p104957}
	\item Laboratorio 10: \url{https://virtuale.unibo.it/mod/forum/discuss.php?d=66753#p106591}
\end{itemize}


\bibliographystyle{alpha}
\bibliography{13-template}