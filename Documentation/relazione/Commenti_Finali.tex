% ------------------------ COMMENTI FINALI---------------------------------------

\chapter{Commenti finali}

\section{Autovalutazione e lavori futuri}

% ------------------------ ALESSANDRO PIOGGIA ----------------------------------

\subsection*{Alessandro Pioggia}

Il seguente progetto mi ha permesso di crescere tanto e sopratutto di riconoscere e individuare gli errori da me commessi.
A mio avviso si è rivelato fondamentale seguire le direttive a noi impartite, specialmente in un lavoro a gruppi, anche un solo metodo non chiamato in maniera adeguata può mettere in seria difficoltà i compagni.
Devo ammettere che non è stato facile gestire le tempistiche, non ho considerato il tempo necessario per il setup del progetto e per eventuali errori(esempio : errori nella gradle build).
\\

\begin{flushleft}
	Punti di forza:
\end{flushleft}

\begin{itemize}
	\item flessibilità, capacità di mettersi in discussione
	\item comunicazione con i compagni
\end{itemize}

\begin{flushleft}
	Punti deboli :
\end{flushleft}

\begin{itemize}
	\item gestione delle tempistiche
	\item mancato uso di programmazione funzionale
	\item ridotta complessità generale
\end{itemize}

% ------------------------ LEON BAIOCCHI ---------------------------------------

\subsection*{Leon Baiocchi}
Questo progetto è stato molto utile per approfondire la conoscenza di me stesso ed imparare a lavorare in gruppo in maniera coesa ed organizzata. Anche se l'organizzazione non è stata il nostro forte, soprattutto nelle fasi iniziali, trovo che tutti siamo maturati molto nel lavoro di gruppo, difatti una volta trovati i nostri punti forti siamo stati in grado di coordinarci meglio su aspetti diversi del progetto in maniera abbastanza complementare. Col senno di poi penso che forse avremmo dovuto concentrarci di più sull'iteratività del processo progettuale.
\textsf{\small }

\begin{flushleft}
	
\textsf{\small Punti di forza:}\\

\begin{itemize}
	\item \textsf{\small Capacità di avere una visione ampia del progetto} 
	\item \textsf{\small Capacità di confrontarsi in maniera oggettiva con i compagni}
	\item \textsf{\small Trasparenza e sincerità}
\end{itemize}

\textsf{\small Punti deboli: }\\

\begin{itemize}
	\item \textsf{\small Rispetto delle tempistiche}
	\item \textsf{\small Potevo utilizzare più streams, soprattutto nella gestione delle entità di gioco/sprites}
\end{itemize}

\end{flushleft}


% ------------------------ FEDERICO BRUNELLI -----------------------------------

\subsection*{Federico Brunelli}

\textsf{\small Questo progetto mi ha dato la possibilità di accresecere le competenze in programmazione e di compredere meglio il significato di progettazione.
Ho potuto capire più a fondo cosa vuol dire lavorare in gruppo, ed è un aspetto fondamentale sia nella progettazione, sia nella fase di sviluppo.
Sono soddisfatto del lavoro compiuto assieme, nonostante il fattore tempo che ha sicuramente influenzato la effettiva realizzazione.}

\begin{flushleft}
	
	\textsf{\small Punti di forza: }\\
	
	\begin{itemize}
		\item \textsf{\small Buona comicazione all'interno del team} 
		\item \textsf{\small Tempestività nella risoluzione di problemi}
		\item \textsf{\small Integrazione dei compiti indivuali all'interno della struttura generale}
	\end{itemize}
	
	\textsf{\small Punti deboli: }\\
	
	\begin{itemize}
		\item \textsf{\small Gestione del tempo non ottimale}
		\item \textsf{\small Superficialità nell'uso di test attraverso JUnit}
		\item \textsf{\small Obiettivi secondari non sviluppati}
	\end{itemize}
	
\end{flushleft}

% ------------------------ LUCA RENGO ------------------------------------------

\subsection*{Luca Rengo}

\textsf{\small Questo primo progetto mi ha sicuramente fatto comprendere meglio l'importanza di una buona organizzazione, gestione delle tempistiche e di quanto sia essenziale una corretta coordinazione e comunicazione del lavoro.}\\
\textsf{\small Alla fine di tutto, ciò che è fondamentale per una produttiva esecuzione del progetto è un effettivo teamwork e workflow. } %determinante

\begin{flushleft}
	
	\textsf{\small Punti di forza:}\\
	
	\begin{itemize}
		\item \textsf{\small Esaminare e discutere i vari aspetti del progetto assieme ai colleghi e cercare di trovare soluzioni comuni sul come affrontarli. }
		\item \textsf{\small Complementarietà del team, ogni membro ha la sua parte specifica del progetto.}
		\item \textsf{\small Risoluzione dei problemi in maniera unita.}
		\item \textsf{\small Scambio di opinioni oneste.}
	\end{itemize}
	
	\textsf{\small Punti deboli: }\\
	
	\begin{itemize}
		\item \textsf{\small Gestione del tempo a disposizione.}
		\item \textsf{\small Poca chiarezza sul da farsi in modo pratico e preciso, dovuto anche al fatto che fosse la prima volta che facevamo un progetto così grande.}
		\item \textsf{\small Realizzazione del lavoro e della sua comunicazione.}
	\end{itemize}
	
\end{flushleft}

\begin{comment}
\textbf{È richiesta una sezione per ciascun membro del gruppo, obbligatoriamente}.
%
Ciascuno dovrà autovalutare il proprio lavoro, elencando i punti di forza e di debolezza in quanto prodotto.
Si dovrà anche cercare di descrivere \emph{in modo quanto più obiettivo possibile} il proprio ruolo all'interno del gruppo.
Si ricorda, a tal proposito, che ciascuno studente è responsabile solo della propria sezione: non è un problema se ci sono opinioni contrastanti, a patto che rispecchino effettivamente l'opinione di chi le scrive.
Nel caso in cui si pensasse di portare avanti il progetto, ad esempio perché effettivamente impiegato, o perché sufficientemente ben riuscito da poter esser usato come dimostrazione di esser capaci progettisti, si descriva brevemente verso che direzione portarlo.
\end{comment}

\begin{comment}
\section{Difficoltà incontrate e commenti per i docenti}

Questa sezione, \textbf{opzionale}, può essere utilizzata per segnalare ai docenti eventuali problemi o difficoltà incontrate nel corso o nello svolgimento del progetto, può essere vista come una seconda possibilità di valutare il corso (dopo quella offerta dalle rilevazioni della didattica) avendo anche conoscenza delle modalità e delle difficoltà collegate all'esame, cosa impossibile da fare usando le valutazioni in aula per ovvie ragioni.
%
È possibile che alcuni dei commenti forniti vengano utilizzati per migliorare il corso in futuro: sebbene non andrà a vostro beneficio, potreste fare un favore ai vostri futuri colleghi.
%
Ovviamente \textit{il contenuto della sezione non impatterà il voto finale}.
\end{comment}